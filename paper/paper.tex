% Options for packages loaded elsewhere
% Options for packages loaded elsewhere
\PassOptionsToPackage{unicode}{hyperref}
\PassOptionsToPackage{hyphens}{url}
\PassOptionsToPackage{dvipsnames,svgnames,x11names}{xcolor}
%
\documentclass[
  10pt,
  letterpaper,
  DIV=11,
  numbers=noendperiod]{scrartcl}
\usepackage{xcolor}
\usepackage[top=1in,left=1in,bottom=1in,right=1in,heightrounded]{geometry}
\usepackage{amsmath,amssymb}
\setcounter{secnumdepth}{5}
\usepackage{iftex}
\ifPDFTeX
  \usepackage[T1]{fontenc}
  \usepackage[utf8]{inputenc}
  \usepackage{textcomp} % provide euro and other symbols
\else % if luatex or xetex
  \usepackage{unicode-math} % this also loads fontspec
  \defaultfontfeatures{Scale=MatchLowercase}
  \defaultfontfeatures[\rmfamily]{Ligatures=TeX,Scale=1}
\fi
\usepackage[]{times}
\ifPDFTeX\else
  % xetex/luatex font selection
\fi
% Use upquote if available, for straight quotes in verbatim environments
\IfFileExists{upquote.sty}{\usepackage{upquote}}{}
\IfFileExists{microtype.sty}{% use microtype if available
  \usepackage[]{microtype}
  \UseMicrotypeSet[protrusion]{basicmath} % disable protrusion for tt fonts
}{}
\usepackage{setspace}
\makeatletter
\@ifundefined{KOMAClassName}{% if non-KOMA class
  \IfFileExists{parskip.sty}{%
    \usepackage{parskip}
  }{% else
    \setlength{\parindent}{0pt}
    \setlength{\parskip}{6pt plus 2pt minus 1pt}}
}{% if KOMA class
  \KOMAoptions{parskip=half}}
\makeatother
% Make \paragraph and \subparagraph free-standing
\makeatletter
\ifx\paragraph\undefined\else
  \let\oldparagraph\paragraph
  \renewcommand{\paragraph}{
    \@ifstar
      \xxxParagraphStar
      \xxxParagraphNoStar
  }
  \newcommand{\xxxParagraphStar}[1]{\oldparagraph*{#1}\mbox{}}
  \newcommand{\xxxParagraphNoStar}[1]{\oldparagraph{#1}\mbox{}}
\fi
\ifx\subparagraph\undefined\else
  \let\oldsubparagraph\subparagraph
  \renewcommand{\subparagraph}{
    \@ifstar
      \xxxSubParagraphStar
      \xxxSubParagraphNoStar
  }
  \newcommand{\xxxSubParagraphStar}[1]{\oldsubparagraph*{#1}\mbox{}}
  \newcommand{\xxxSubParagraphNoStar}[1]{\oldsubparagraph{#1}\mbox{}}
\fi
\makeatother

\usepackage{color}
\usepackage{fancyvrb}
\newcommand{\VerbBar}{|}
\newcommand{\VERB}{\Verb[commandchars=\\\{\}]}
\DefineVerbatimEnvironment{Highlighting}{Verbatim}{commandchars=\\\{\}}
% Add ',fontsize=\small' for more characters per line
\usepackage{framed}
\definecolor{shadecolor}{RGB}{241,243,245}
\newenvironment{Shaded}{\begin{snugshade}}{\end{snugshade}}
\newcommand{\AlertTok}[1]{\textcolor[rgb]{0.68,0.00,0.00}{#1}}
\newcommand{\AnnotationTok}[1]{\textcolor[rgb]{0.37,0.37,0.37}{#1}}
\newcommand{\AttributeTok}[1]{\textcolor[rgb]{0.40,0.45,0.13}{#1}}
\newcommand{\BaseNTok}[1]{\textcolor[rgb]{0.68,0.00,0.00}{#1}}
\newcommand{\BuiltInTok}[1]{\textcolor[rgb]{0.00,0.23,0.31}{#1}}
\newcommand{\CharTok}[1]{\textcolor[rgb]{0.13,0.47,0.30}{#1}}
\newcommand{\CommentTok}[1]{\textcolor[rgb]{0.37,0.37,0.37}{#1}}
\newcommand{\CommentVarTok}[1]{\textcolor[rgb]{0.37,0.37,0.37}{\textit{#1}}}
\newcommand{\ConstantTok}[1]{\textcolor[rgb]{0.56,0.35,0.01}{#1}}
\newcommand{\ControlFlowTok}[1]{\textcolor[rgb]{0.00,0.23,0.31}{\textbf{#1}}}
\newcommand{\DataTypeTok}[1]{\textcolor[rgb]{0.68,0.00,0.00}{#1}}
\newcommand{\DecValTok}[1]{\textcolor[rgb]{0.68,0.00,0.00}{#1}}
\newcommand{\DocumentationTok}[1]{\textcolor[rgb]{0.37,0.37,0.37}{\textit{#1}}}
\newcommand{\ErrorTok}[1]{\textcolor[rgb]{0.68,0.00,0.00}{#1}}
\newcommand{\ExtensionTok}[1]{\textcolor[rgb]{0.00,0.23,0.31}{#1}}
\newcommand{\FloatTok}[1]{\textcolor[rgb]{0.68,0.00,0.00}{#1}}
\newcommand{\FunctionTok}[1]{\textcolor[rgb]{0.28,0.35,0.67}{#1}}
\newcommand{\ImportTok}[1]{\textcolor[rgb]{0.00,0.46,0.62}{#1}}
\newcommand{\InformationTok}[1]{\textcolor[rgb]{0.37,0.37,0.37}{#1}}
\newcommand{\KeywordTok}[1]{\textcolor[rgb]{0.00,0.23,0.31}{\textbf{#1}}}
\newcommand{\NormalTok}[1]{\textcolor[rgb]{0.00,0.23,0.31}{#1}}
\newcommand{\OperatorTok}[1]{\textcolor[rgb]{0.37,0.37,0.37}{#1}}
\newcommand{\OtherTok}[1]{\textcolor[rgb]{0.00,0.23,0.31}{#1}}
\newcommand{\PreprocessorTok}[1]{\textcolor[rgb]{0.68,0.00,0.00}{#1}}
\newcommand{\RegionMarkerTok}[1]{\textcolor[rgb]{0.00,0.23,0.31}{#1}}
\newcommand{\SpecialCharTok}[1]{\textcolor[rgb]{0.37,0.37,0.37}{#1}}
\newcommand{\SpecialStringTok}[1]{\textcolor[rgb]{0.13,0.47,0.30}{#1}}
\newcommand{\StringTok}[1]{\textcolor[rgb]{0.13,0.47,0.30}{#1}}
\newcommand{\VariableTok}[1]{\textcolor[rgb]{0.07,0.07,0.07}{#1}}
\newcommand{\VerbatimStringTok}[1]{\textcolor[rgb]{0.13,0.47,0.30}{#1}}
\newcommand{\WarningTok}[1]{\textcolor[rgb]{0.37,0.37,0.37}{\textit{#1}}}

\usepackage{longtable,booktabs,array}
\usepackage{calc} % for calculating minipage widths
% Correct order of tables after \paragraph or \subparagraph
\usepackage{etoolbox}
\makeatletter
\patchcmd\longtable{\par}{\if@noskipsec\mbox{}\fi\par}{}{}
\makeatother
% Allow footnotes in longtable head/foot
\IfFileExists{footnotehyper.sty}{\usepackage{footnotehyper}}{\usepackage{footnote}}
\makesavenoteenv{longtable}
\usepackage{graphicx}
\makeatletter
\newsavebox\pandoc@box
\newcommand*\pandocbounded[1]{% scales image to fit in text height/width
  \sbox\pandoc@box{#1}%
  \Gscale@div\@tempa{\textheight}{\dimexpr\ht\pandoc@box+\dp\pandoc@box\relax}%
  \Gscale@div\@tempb{\linewidth}{\wd\pandoc@box}%
  \ifdim\@tempb\p@<\@tempa\p@\let\@tempa\@tempb\fi% select the smaller of both
  \ifdim\@tempa\p@<\p@\scalebox{\@tempa}{\usebox\pandoc@box}%
  \else\usebox{\pandoc@box}%
  \fi%
}
% Set default figure placement to htbp
\def\fps@figure{htbp}
\makeatother





\setlength{\emergencystretch}{3em} % prevent overfull lines

\providecommand{\tightlist}{%
  \setlength{\itemsep}{0pt}\setlength{\parskip}{0pt}}



 
\usepackage[style=apa,maxbibnames=99,dashed=false,sorting=ydnt]{biblatex}
\addbibresource{bibliography.bib}


\usepackage{gb4e}
\noautomath
% \usepackage[inline]{glossaries}
\usepackage{leipzig}
% \makeglossaries
\usepackage{typgloss}
\usepackage{setspace}
\usepackage{lineno}
\linenumbers
\KOMAoption{captions}{tableheading}
\makeatletter
\@ifpackageloaded{caption}{}{\usepackage{caption}}
\AtBeginDocument{%
\ifdefined\contentsname
  \renewcommand*\contentsname{Table of contents}
\else
  \newcommand\contentsname{Table of contents}
\fi
\ifdefined\listfigurename
  \renewcommand*\listfigurename{List of Figures}
\else
  \newcommand\listfigurename{List of Figures}
\fi
\ifdefined\listtablename
  \renewcommand*\listtablename{List of Tables}
\else
  \newcommand\listtablename{List of Tables}
\fi
\ifdefined\figurename
  \renewcommand*\figurename{Figure}
\else
  \newcommand\figurename{Figure}
\fi
\ifdefined\tablename
  \renewcommand*\tablename{Table}
\else
  \newcommand\tablename{Table}
\fi
}
\@ifpackageloaded{float}{}{\usepackage{float}}
\floatstyle{ruled}
\@ifundefined{c@chapter}{\newfloat{codelisting}{h}{lop}}{\newfloat{codelisting}{h}{lop}[chapter]}
\floatname{codelisting}{Listing}
\newcommand*\listoflistings{\listof{codelisting}{List of Listings}}
\makeatother
\makeatletter
\makeatother
\makeatletter
\@ifpackageloaded{caption}{}{\usepackage{caption}}
\@ifpackageloaded{subcaption}{}{\usepackage{subcaption}}
\makeatother
\usepackage{bookmark}
\IfFileExists{xurl.sty}{\usepackage{xurl}}{} % add URL line breaks if available
\urlstyle{same}
\hypersetup{
  pdftitle={Agreement attraction under uncertainty: A case of register manipulation},
  pdfauthor={Utku Turk},
  colorlinks=true,
  linkcolor={blue},
  filecolor={Maroon},
  citecolor={Blue},
  urlcolor={Blue},
  pdfcreator={LaTeX via pandoc}}


\title{Agreement attraction under uncertainty: A case of register
manipulation}
\author{Utku Turk}
\date{}
\begin{document}
\maketitle
\begin{abstract}
Speakers often make systematic errors in establishing a number agreement
relation between a verb and its agreement controller, when another NP
with a different number (the attractor) interferes. As a result,
speakers may produce ungrammatical sentences like '*The key to the
cabinets are rusty,' or misclassify them as acceptable (Bock and Miller
1991). However, it has been noted that these judgment errors are done
predominantly in ungrammatical sentences and speakers do not misclassify
grammatical sentences like `The key to the cabinets is rusty,' as
ungrammatical due to the presence of the attractor. This grammaticality
asymmetry has been taken to support a specific account that utilizes
privative features and reanalysis. According to these accounts (Lewis
and Vasishth 2005, Wagers et al.~2009), this phenomenon, called
attraction, arises as a result of a reanalysis of the attractor as the
agreement controller at the site of the verb. As for grammatical
sentences, since no reanalysis is needed, speakers do not make similar
errors. However, the true extent of this phenomenon cannot be correctly
measured in grammatical sentences, in which the overall accuracy is
close to `ceiling' (Uttl 2005). Recently, it was shown that when
participants' a priori response bias towards `yes' is manipulated and
the overall accuracy in grammatical sentences is lowered, the similar
effects arise in both grammatical and ungrammatical sentences (Hammerly
et al.~2019, Turk 2022). However, these experiments manipulated the
response bias through instructions and ratio of ungrammatical sentences
to grammatical sentences. This work aims to investigate a more
naturalistic approach to this question by utilizing the effects of
register on agreement in Turkish, in which attraction was previously
attested as well (Lago et al.~2019, Turk and Logacev 2024). Turkish
register facts let us exploit two complementary manipulations. First, an
overt formal addressee (e.g., `sir/efendim') can control agreement,
rendering an otherwise ungrammatical string acceptable ( = `The key to
the cabinets are rusty, sir'). Second, an overt informal addressee
uniformly decreases acceptability (= `\#The key to the cabinet is rusty,
yo'), thereby removing ceiling effects on grammatical baselines. Using
these register cues in a speeded AJT (N=174), we observe attraction
illusions in grammatical sentences once ceilings are reduced, producing
parallel effects across grammatical and ungrammatical items. This
symmetry undermines accounts that tie attraction solely to reanalysis
under a privative-feature retrieval mechanism.
\end{abstract}


\setstretch{1}
\begin{center}\rule{0.5\linewidth}{0.5pt}\end{center}

\section{Introduction}\label{introduction}

\begin{Shaded}
\begin{Highlighting}[]

\DecValTok{2} \SpecialCharTok{+} \DecValTok{2} 
\end{Highlighting}
\end{Shaded}

\subsection{Phenomenon: Agreement
attraction}\label{phenomenon-agreement-attraction}

Agreement attraction occurs when a verb agrees with a nearby noun phrase
(the attractor) instead of the true subject. In Turkish possessive DPs
like \emph{yöneticilerin aşçısı} (``the managers' cook''), speakers
sometimes accept sentences where the verb agrees with the plural
possessor rather than the singular head. The basic behavioral signature
is that ungrammatical sentences with a plural attractor receive more
``yes/acceptable'' responses than when the attractor is singular
\autocite{Lago2019,Turk2022,Ulusoy2023,TurkLogacev2024a}. In grammatical
sentences, however, attraction is often absent or highly attenuated.

\subsection{Competing accounts}\label{competing-accounts}

Two families of accounts aim to explain this illusion:

\begin{itemize}
\tightlist
\item
  \textbf{Retrieval-based}: Cue-based retrieval models argue that the
  verb initiates a search for a controller, and partial matches
  sometimes satisfy the retrieval process
  \autocite{Wagers2009,Eberhard2005,Engelmann2019}. Predicts strong
  attraction in ungrammaticals, but little or none in grammatical
  sentences.
\item
  \textbf{Representational distortion}: Feature spreading or
  number-marking ``distortion'' can blur DP representations, producing
  hybrid subjects that yield attraction even in grammatical sentences
  \autocite{Hammerly2019,Yadav2023}.
\end{itemize}

\subsection{The asymmetry puzzle}\label{the-asymmetry-puzzle}

Attraction is consistently robust in ungrammaticals but weak or absent
in grammaticals across many languages
\autocite{BockMiller1991,Wagers2009}. This asymmetry has been used as
support for retrieval accounts. But it may also reflect task artifacts
like ceiling effects or response bias
\autocite{Hammerly2019,TurkLogacev2023}.

\subsection{Prior attempts to address the
asymmetry}\label{prior-attempts-to-address-the-asymmetry}

\begin{itemize}
\tightlist
\item
  \textbf{Between-subject manipulation}: Instruction and ratio
  manipulations reduced ``yes'' bias \autocite{Hammerly2019} but were
  difficult to implement.\\
\item
  \textbf{Bias grouping}: Post-hoc grouping by filler-based bias
  reproduced attraction in grammaticals for some participants
  \autocite{TurkLogacev2023}, but did not generalize across experiments.
\end{itemize}

\subsection{Aim of the present study}\label{aim-of-the-present-study}

We test whether attraction in grammatical sentences can be observed
\textbf{within-subjects}, using Turkish register manipulations to
modulate baseline acceptability and reveal hidden effects.

\section{Current Study}\label{current-study}

\subsection{Turkish register facts
leveraged}\label{turkish-register-facts-leveraged}

\subsubsection{Formal addressee}\label{formal-addressee}

An explicit formal addressee (e.g., \emph{efendim}) licenses plural
agreement on the verb, independent of subject number
\autocite{Turk2022,TurkLogacev2024b}.\\
- Raises acceptability, introduces an alternative controller, and
maintains ceiling effects.

\subsubsection{Informal addressee with hierarchical
nouns}\label{informal-addressee-with-hierarchical-nouns}

Informal addressees (e.g., \emph{lan/yo}) do not license agreement.
Alone, they have no effect.\\
- When combined with hierarchical nouns, they induce register
incongruence, lowering acceptability \autocite{Ulusoy2023}.\\
- This creates space for attraction in grammatical sentences.

\subsection{Experimental design}\label{experimental-design}

\subsubsection{Task}\label{task}

\begin{itemize}
\tightlist
\item
  Speeded acceptability judgment.\\
\item
  Word-by-word presentation at fixed rate, followed by a brief judgment
  window.\\
\item
  Response: binary yes/no to whether the sentence was acceptable.
\end{itemize}

\subsubsection{Factors}\label{factors}

\begin{itemize}
\tightlist
\item
  Attractor number: singular vs plural.\\
\item
  Verb number: singular vs plural (defines grammaticality).\\
\item
  Register: formal vs informal.\\
\item
  All manipulated within participants.
\end{itemize}

\subsubsection{Materials}\label{materials}

\begin{itemize}
\tightlist
\item
  Based on Lago et al.'s Turkish attraction items \autocite{Lago2019}.\\
\item
  Each sentence appears in one of eight conditions (2 × 2 × 2).\\
\item
  Latin-square assignment so that each participant sees each lexical
  item only once.\\
\item
  Formal versions include \emph{efendim}; informal versions include
  \emph{lan/yo}.
\end{itemize}

\subsubsection{Participants and fillers}\label{participants-and-fillers}

\begin{itemize}
\tightlist
\item
  174 participants.\\
\item
  48 critical items, 96 fillers.\\
\item
  Equal proportion of grammatical and ungrammatical fillers.\\
\item
  Same instructions as earlier Turkish attraction experiments to keep
  comparability.
\end{itemize}

\subsection{Predictions}\label{predictions}

\subsubsection{Shared across theories}\label{shared-across-theories}

\begin{itemize}
\tightlist
\item
  Plural attractor should increase ``yes'' rates in ungrammaticals.
\end{itemize}

\subsubsection{Retrieval account}\label{retrieval-account}

\begin{itemize}
\tightlist
\item
  No effect of attractor number in grammaticals, because the correct
  singular subject is a perfect match.
\end{itemize}

\subsubsection{Representational account}\label{representational-account}

\begin{itemize}
\tightlist
\item
  Attraction in both grammatical and ungrammatical sentences, because
  the subject representation itself can be distorted by the plural
  possessor.
\end{itemize}

\subsubsection{Register-based
predictions}\label{register-based-predictions}

\begin{itemize}
\tightlist
\item
  Formal addressee: increases acceptability across the board; ceiling
  remains; attraction may still be hidden in grammaticals.\\
\item
  Informal+hierarchy: decreases acceptability of grammatical sentences;
  ceiling is removed; attraction should emerge even in grammaticals.
\end{itemize}

\subsection{Results}\label{results}

\subsubsection{Formal register}\label{formal-register}

\begin{itemize}
\tightlist
\item
  Ungrammaticals: robust attraction (plural attractor → more ``yes'').\\
\item
  Grammatically well-formed sentences: no attraction; ceiling effect
  persists.
\end{itemize}

\subsubsection{Informal register}\label{informal-register}

\begin{itemize}
\tightlist
\item
  Ungrammaticals: robust attraction.\\
\item
  Grammatically well-formed sentences: attraction appears; plural
  attractor lowers ``yes'' judgments compared to singular attractor.\\
\item
  Example: grammatical sentences judged ungrammatical
  \textasciitilde20\% of the time when informal addressee and plural
  possessor combined.
\end{itemize}

\subsection{Modeling results}\label{modeling-results}

\begin{itemize}
\tightlist
\item
  Outcome: ``yes'' response (Bernoulli).\\
\item
  Predictors: Verb Number × Attractor Number × Register, with trial
  order.\\
\item
  Random intercepts and slopes for subjects and items
  \autocite{Barr2013}.\\
\item
  Bayesian hierarchical logistic regression
  \autocite{GelmanHill2007,NicenboimVasishth2016,Kruschke2018}.
\end{itemize}

Key estimates: - Plural verb → reduces ``yes'' responses overall.\\
- Plural attractor → increases ``yes'' responses overall, especially
with plural verbs.\\
- Formal register → raises ``yes'' responses overall.\\
- Crucial three-way interaction: informal register shows reduced
asymmetry, i.e., attraction in grammaticals emerges only under informal
conditions.

\subsection{Interim inference}\label{interim-inference}

The asymmetry between ungrammaticals and grammaticals is not a
structural fact about agreement computation. Instead, it reflects
task-level ceiling effects. By lowering the ceiling through register
incongruence, attraction can be revealed in grammatical sentences.

\section{Discussion}\label{discussion}

\subsection{What the register manipulation
demonstrates}\label{what-the-register-manipulation-demonstrates}

\begin{itemize}
\tightlist
\item
  Grammaticality asymmetry in attraction is not intrinsic to the
  agreement system.\\
\item
  Task and register manipulations can shape whether attraction effects
  surface in grammaticals.\\
\item
  Register influences the ``acceptability space'' in which attraction
  can be measured.
\end{itemize}

\subsection{Implications for retrieval
accounts}\label{implications-for-retrieval-accounts}

\begin{itemize}
\tightlist
\item
  Retrieval models remain viable: when the subject is a perfect match,
  attraction can be hidden under ceiling conditions
  \autocite{Wagers2009,Engelmann2019}.\\
\item
  However, when register increases uncertainty about the controller,
  retrieval is more vulnerable to interference.\\
\item
  Suggests that retrieval dynamics depend on both structural cues and
  contextual uncertainty.
\end{itemize}

\subsection{Implications for representational
accounts}\label{implications-for-representational-accounts}

\begin{itemize}
\tightlist
\item
  Evidence that attraction can appear in grammaticals supports the idea
  that distorted representations sometimes affect judgment
  \autocite{Yadav2023}.\\
\item
  However, the effect is conditional: it surfaces only when ceiling is
  reduced.\\
\item
  Thus representational distortion may modulate retrieval, rather than
  fully replacing it.
\end{itemize}

\subsection{Bias versus computation}\label{bias-versus-computation}

\begin{itemize}
\tightlist
\item
  Earlier ``yes-bias'' manipulations showed that participant-level bias
  shapes attraction asymmetry \autocite{Hammerly2019,TurkLogacev2023}.\\
\item
  The current register-based manipulation shows the same shift can occur
  \textbf{within subjects}, without between-subject designs or post-hoc
  grouping.
\end{itemize}

\subsection{Limitations and future
directions}\label{limitations-and-future-directions}

\begin{itemize}
\tightlist
\item
  Formal vs informal manipulations tested only with possessor-head DPs
  and addressee expressions.\\
\item
  Future work:

  \begin{itemize}
  \tightlist
  \item
    Vary position of addressee within the sentence.\\
  \item
    Extend to other structures beyond possessives.\\
  \item
    Explore dialectal variation in register-controlled agreement.\\
  \item
    Test in production tasks and comprehension measures (e.g., reading
    times, ERPs).
  \end{itemize}
\end{itemize}

\subsection{Broader connections}\label{broader-connections}

\begin{itemize}
\tightlist
\item
  Bias in acceptability judgments
  \autocite{Hammerly2019,TurkLogacev2023}.\\
\item
  Case syncretism effects in Turkish: attraction not modulated by
  subject case \autocite{TurkLogacev2024a}.\\
\item
  Planning agreement independent of verb planning in production
  \autocite{TurkLauPhillips2025}.\\
\item
  Contributes to a growing picture in which attraction is shaped by both
  structural representation and task-level pressures.
\end{itemize}

\subsection{Open questions}\label{open-questions}

\begin{itemize}
\tightlist
\item
  How much reduction in acceptability is required for
  grammatical-sentence attraction to appear?\\
\item
  Are register-based manipulations equivalent to instruction-based
  manipulations, or do they operate differently?\\
\item
  How do addressee-triggered plural effects interact with other features
  (e.g., person marking)?\\
\item
  Can similar manipulations reveal attraction in comprehension tasks, or
  is this specific to judgment paradigms?
\end{itemize}

\section{References}\label{references}

\printbibliography[heading=none]





\end{document}
